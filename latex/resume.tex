%-------------------------
% Author : Tsou, Dong-You
%-------------------------



%------------PACKAGES----------------
\documentclass[a4paper,10.5pt]{article}

\usepackage{verbatim} % reimplements the "verbatim" and "verbatim*" environments

\usepackage{titlesec} % provides an interface to sectioning commands i.e. custom elements

\usepackage{color} % provides both foreground and background color management

\usepackage{enumitem} % provides control over enumerate, itemize and description

\usepackage{fancyhdr} % provides extensive facilities for constructing headers, footers and also controlling their use

\usepackage{tabularx} % defines an environment tabularx, extension of "tabular" with an extra designator x, paragraph like column whose width automatically expands to fill the width of the environment

\usepackage{latexsym} % provides mathematical symbols

\usepackage{marvosym} % provides martin vogel's symbol font which contains various symbols

\usepackage[empty]{fullpage} % sets margins to one inch and removes headers, footers etc..

\usepackage[hidelinks]{hyperref} % removes color and shadow of hyperlinks

\usepackage[normalem]{ulem} % provides "\ul" (uline) command which will break at line breaks

\usepackage[english]{babel} % provides culturally determined typographical rules for wide range of languages
%-----------------------------------------

\input glyphtounicode % converts glyph names to unicode
\pdfgentounicode=1 % ensures pdfs generated are ats readable

%----------FONT OPTIONS-------------------
\usepackage[default]{sourcesanspro} % uses the font source sans pro
\urlstyle{same} % changes url font from default urlfont to font being used by the document
%-----------------------------------------


%----------MARGIN OPTIONS-----------------
\pagestyle{fancy} % set page style to one configured by fancyhdr
\fancyhf{} % clear all header and footer fields

\renewcommand{\headrulewidth}{0in} % sets thickness of linerule under header to zero
\renewcommand{\footrulewidth}{0in} % sets thickness of linerule over footer to zero

\setlength{\tabcolsep}{0in} % sets thickness of column separator in tables to zero

% origin of the document is one inch from the top and from and the left
% oddsidemargin and evensidemargin both refer to the left margin
% right margin is indirectly set using oddsidemargin
\addtolength{\oddsidemargin}{-0.7in}
\addtolength{\topmargin}{-0.5in}

\addtolength{\textwidth}{1in} % sets width of text area in the page to 1.3 inch
\addtolength{\textheight}{2in} % sets height of text area in the page to 0.5 inch

\raggedbottom{} % makes all pages the height of current page, no extra vertical space added
\raggedright{} % makes all pages the width of current page, no extra horizontal space added
%------------------------------------------


%--------SECTIONING COMMANDS---------
% \titleformat{<command>}
%   [<shape>]{<format>}{<label>}{<sep>}
%     {<before-code>}[<after-code>]

% command is the sectioning command to be redefined
% shape is the style of the font; scshape stands for small caps style
% format is the format to be applied to whole title- label and text; absent here
% label defines the label
% sep is the horizontal separation between label and title body
% before-code is the code to be executed before
% after-code is the code to be executed after

\titleformat{\section}
  {\scshape\large}{}
    {0em}{\color{blue}}[\color{black}\titlerule\vspace{-2pt}]
%-------------------------------------


%--------REDEFINITIONS----------------
% redefines the style of the bullet point
\renewcommand\labelitemii{$\vcenter{\hbox{\tiny$\bullet$}}$}

% redefines the underline depth to 1pt
\renewcommand{\ULdepth}{1pt}
%-------------------------------------


%--------CUSTOM COMMANDS--------------
%\vspace{} defines a vertical space of given size, modifying this in custom commands can help stretch or shrink resume to remove or add content

% resumeItem renders a bullet point
\newcommand{\resumeItem}[1]{
  \item\small{#1}
}

% commands to start and end itemization of resumeItem, rightmargin set to 0.11in to avoid the overflow of resumetItem beyond whatever resumeItemHeading is being used
\newcommand{\resumeItemListStart}{\begin{itemize}[rightmargin=0.1in]}
\newcommand{\resumeItemListEnd}{\end{itemize}}

% resumeSectionType renders a bolded type to be used under a section, used as skill type here, middle element is used to keep ":"s in the same vertical line
\newcommand{\resumeSectionType}[3]{
  \item\begin{tabular*}{0.95\textwidth}[t]{
    p{0.23\linewidth}p{0.01\linewidth}p{0.74\linewidth}
  }
    \textbf{#1} & #2 & #3
  \end{tabular*}\vspace{-3pt}
}

% resumeTrioHeading renders three elements in three columns with second element being italicized and first element bolded, can be used for projects with three elements
\newcommand{\resumeTrioHeading}[3]{%
  \item\small{%
    \makebox[\linewidth]{%
      \makebox[0.33\linewidth][l]{\textbf{#1} }%
      \makebox[0.34\linewidth][c]{\textit{#2} }%
      \makebox[0.33\linewidth][r]{#3 }%
    }%
  }%
}


% resumeQuadHeading renders four elements in a two columns with the second row being italicized and first element of first row bolded, can be used for experience and projects with four elements
\newcommand{\resumeQuadHeading}[4]{
  \item
  \begin{tabular*}{0.96\textwidth}[t]{l@{\extracolsep{\fill}}r}
    \textbf{#1} & #2 \\
    \textit{\small#3} & \textit{\small #4} \\
  \end{tabular*}
}

% resumeQuadHeadingDetails renders three elements in a two columns with the second row being italicized and first element of first row bolded, can be used for experience and projects.
\newcommand{\resumeQuadHeadingDetails}[3]{%
   \item
  \begin{tabular*}{0.96\textwidth}[t]{l@{\extracolsep{\fill}}r}
    \textbf{#1} & #2 \\
    \textit{\small #3} & \\
  \end{tabular*}
}

% resumeQuadHeadingChild renders the second row of resumeQuadHeading, can be used for experience if different roles in the same company need to added
\newcommand{\resumeQuadHeadingChild}[2]{
  \item
  \begin{tabular*}{0.96\textwidth}[t]{l@{\extracolsep{\fill}}r}
    \textbf{\small#1} & {\small#2} \\
  \end{tabular*}
}

% commands to start and end itemization of resumeQuadHeading, lefmargin for left indent of 0.15in for resumeItems
\newcommand{\resumeHeadingListStart}{
  \begin{itemize}[leftmargin=0.15in, label={}]
}
\newcommand{\resumeHeadingListEnd}{\end{itemize}}
%-------------------------------------------


%__________________RESUME____________________
% You can rearrange sections in any order you may prefer
\begin{document}

%-----------CONTACT DETAILS------------------
% Make sure all the details are correct, you can add more links in the first row of second column if needed

\begin{tabular*}{\textwidth}{l@{\extracolsep{\fill}}r}
  \textbf{\Huge Tsou, Dong-You \vspace{2pt}}  \\ % row = 1, col = 2
  \href{https://www.linkedin.com/in/dytsou/}{\uline{LinkedIn}} $|$ % row = 2, col = 1
  \href{https://github.com/dytsou/}{\uline{GitHub}}  % row = 2, col = 1
   &
  Email: \href{mailto:contact@dy.tsou.me}{\uline{contact@dy.tsou.me}} $|$ % row = 2, col = 2
  Mobile: +886981618820 \\ % row = 2, col = 2
\end{tabular*}
%--------------------------------------------

%-----------EDUCATION-------------------------
% Mention your CGPA, if its good, in the first row of second column

\section{Education}
  \resumeHeadingListStart{}
    \resumeQuadHeading{National Yang Ming Chiao Tung University}{Hsinchu, Taiwan}
    {Bachelor of Science in Computer Science}{Sept. 2022 -- Jun. 2026}
  \resumeHeadingListEnd{}
%---------------------------------------------

%--------------SKILLS------------------------
% Add or remove resumeSectionTypes according to your needs

\section{Technical Skills}
  \resumeHeadingListStart{}
    \resumeSectionType{Languages}{}{ Python, C++, Go, HTML, CSS, JavaScript, Dart, PHP, Verilog, LabVIEW}
    \resumeSectionType{Frameworks \& Libraries}{}{React.js, Node.js, Flutter,  Flask, FastAPI, OpenGL}
    \resumeSectionType{Tools \& Technologies}{}{Git, PostgreSQL, Arduino, Docker, Linux, API Development, RESTful Services}
    \resumeSectionType{Specialized Skills}{}{Web Development: Full-stack development with React frontend and backend APIs}
    \resumeSectionType{}{}{Game Development: OpenGL-based 3D graphics and LabVIEW multiplayer systems}
    % \resumeSectionType{}{}{Bot Development: LINE bot applications with Python}
    \resumeSectionType{}{}{Computer Vision: Real-time 3D object classification in ground (AV) and aerial (UAV) scenarios}
  \resumeHeadingListEnd{}
%--------------------------------------------

% -----------LABORATORY EXPERIENCE-----------------------

\section{Laboratory Experience}
\resumeHeadingListStart{}
  \resumeQuadHeadingDetails{\href{https://sqlab.web.nycu.edu.tw}{Software Quality Lab}}{Sept. 2025 -- Present}
  {The Influence of AI-Generated Testing on Developer Productivity}
    \resumeItemListStart{}
      \resumeItem{Integrated AI-assisted test generation into development workflows to achieve measurable gains in efficiency and coverage.}
      \resumeItem{Researched the impact of AI-generated testing on developer productivity and software quality, with a focus on quantitative results.}
    \resumeItemListEnd{}

  \resumeQuadHeadingDetails{\href{http://acm.cs.nycu.edu.tw/}{Applied Computing and Multimedia Lab}}{Sept. 2024 -- Aug. 2025}
  {Video-based 3D Object Detection with Vision Foundation Models}
    \resumeItemListStart{}
      \resumeItem{Built a monocular video 3D detector with SAM 2 for temporally consistent segmentation and depth-aware boxes}
      \resumeItem{Applied LoRA and BEVDepth-style depth-context fusion for efficient training and temporal box refinement}
    \resumeItemListEnd{}
\resumeHeadingListEnd{}
% ---------------------------------------------

%-----------PROJECTS--------------------------


\section{Projects}
  \resumeHeadingListStart{}

    \resumeTrioHeading{SDC Core System}{Go, RESTful API, Docker, PostgerSQL}{\href{https://github.com/NYCU-SDC/core-system-backend}{\uline{Source Code}}}
      \resumeItemListStart{}
        \resumeItem{Designed and implemented a robust backend infrastructure for NYCU Software Development Club, facilitating cross-unit operations and continuous service availability}
        \resumeItem{Designed modular RESTful APIs and robust authentication, with over 500 code commits}
        \resumeItem{Streamlined CI/CD pipeline using Docker, reducing deployment time}
        \resumeItem{Integrated PostgreSQL databases and multi-environment configs for scalable, maintainable infrastructure}
        \resumeItemListEnd{}

    \resumeTrioHeading{CAIender}{React.js, Vite, GraphQL, DynamoDB, LLM}{\href{https://github.com/orgs/MCHackathon2025/repositories}{\uline{Source Code}}}
      \resumeItemListStart{}
        \resumeItem{Independently designed and developed a prototype for an AI calendar platform based on React architecture}
        \resumeItem{Integrated GraphQL and DynamoDB, leveraging LLM to enable context-aware recommendations and AI scheduling}
        \resumeItem{Implemented Google Map redirection and GPS-based location search to improve user activity discovery}
        \resumeItem{Adopted Vite for build optimization, resulting in faster frontend compilation and enhanced modular scalability}
      \resumeItemListEnd{}

  \resumeHeadingListEnd{}
%--------------------------------------------

%----------------Extracurricular Experiences----------------------

\section{Extracurricular Experiences}
  \resumeHeadingListStart{}
    \resumeQuadHeadingDetails{\href{https://sdc.nycu.club}{NYCU Software Development Club}}{Oct. 2023 -- Present}
  {Vice President}
    \resumeItemListStart{}
            \resumeItem{Coordinated 18 programs across two semesters, supporting a community of 163 members from 42 academic departments and enhancing event quality.}
            \resumeItem{Developed and maintained SDC \href{https://github.com/NYCU-SDC/core-system-backend}{Core System} for program and event management, supporting member tracking, team matching, form workflows, and data analytics to streamline operations and enhance user experience.}
      \resumeItemListEnd{}
  \resumeQuadHeadingDetails{\href{https://sitcon.org/2025/}{SITCON, Students' Information Technology Conference}}{Oct. 2024 -- Mar. 2025}
  {Agenda Committee}
        \resumeItemListStart{}
            \resumeItem{Organizing 28 sessions and coordinating over 2,180 minutes of student rehearsals for 1,400+ participants at Taiwan’s leading student tech conference.}
      \resumeItemListEnd{}
  \resumeHeadingListEnd{}
%--------------------------------------------

\end{document}